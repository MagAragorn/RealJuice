\section*{Задача 35}

Дайте определения точек локального экстремума. Сформулируйте и докажите
теорему Ферма.

\subsection*{Решение}
\begin{defi}
    Пусть $D$ -- область определения $f$, тогда точка $x_0 \in D$
    называется локальным минимумом (максимумом) функции $f(x)$, если
    $$
        [\exists u_{\delta}(x_0)]\big(\forall x \in u_{\delta} \, f(x_0)
        \leq f(x) \big)
    $$
    Для максимума аналогично и $f(x_0) \geq f(x)$.
\end{defi}

\begin{defi}
    Точки локального минимума или максимума называются точками локального
    экстремума.
\end{defi}

\begin{ther}[\textbf{Ферма}]
    Если $f(x)$ имеет локальный экстремум в точке $x_0$ и при этом $\exists
    f\drv(x_0)$, то $f\drv(x_0) = 0$.
\end{ther}

\begin{proof}
    Если существует производная в точке $x_0$, то существуют левая и правая
    производные в этой точке, и они равны. Вычислим их по определению:\par
    $f_{-}\drv(x_0) = \lim\limits_{x \to x_0 - 0}\frac{f(x)-f(x_0}{x-x_0}$;
    поскольку в окрестности $-0$
    $
    \begin{cases}
        f(x) = f(x_0) < 0 \\
        x - x_0 < 0
    \end{cases}
    $, получим $f_{-}\drv(x_0) \geq 0$;
    Аналогично для правой производной: в окрестности $+0$
    $
    \begin{cases}
        f(x) = f(x_0) < 0 \\
        x - x_0 > 0
    \end{cases}
    $, получим $f_{+}\drv(x_0) \leq 0$;\par
    При этом они равны: $f_{+}\drv(x_0) = f_{-}\drv(x_0) \Rightarrow f_{-}\drv(x_0) =
    f_{+}\drv(x_0) = 0$
\end{proof}

