\documentclass[a4paper,12pt]{article}

\usepackage[T1,T2A]{fontenc}        % Кодировки шрифтов
\usepackage[utf8]{inputenc}         % Кодировка текста
\usepackage[english,russian]{babel} % Подключение поддержки языков

\usepackage{amsthm}                 % Оформление теорем
\usepackage{amstext}                % Текстовые вставки в формулы
\usepackage{amsfonts}               % Математические шрифты

\newtheorem*{ther}{Теорема}
\newtheorem*{defi}{Определение}

\newcommand{\eps}{\varepsilon}

\begin{document}

    \section*{Билет 10}
    
    Дайте определения бесконечно малой и бесконечно большой последовательностей и сформулируйте их свойства. Сформулируйте арифметические свойства пределов и докажите их (одно на
выбор экзаменатора). Найдите $\displaystyle \lim_{n\rightarrow \infty} 
            \frac{
                   \sqrt[5]{2n^4 + 3} \cdot 
                   \sqrt[7]{3n^3 - 1}
                 }
                 {
                   \sqrt[15]{7n^{18} + 3} + 
                   \sqrt[3]{4n^4 + 1}
                 }.$
    
    \subsection*{Решение}
    
    \begin{defi} Последовательность $ \{ x_ n \} $ называется бесконечно малой, 
        если $\displaystyle \lim_{n \rightarrow \infty} x_n = 0$, т.е. 
        $\forall \eps > 0\ \exists N = N(\eps):|x_n| < \eps\ \forall n > N(\eps).$
    \end{defi}

    \textbf{Свойства б.м. последовательностей:}
        \begin{enumerate}
            \item $\exists \lim y_n = a \Leftrightarrow \{x_n = y_n - a\}$ -- бесконечно малая.
            \item Сумма \textbf{конечного} числа бесконечно малых посл-ей -- есть бесконечно малая посл-ть.
            \item Сумма бесконечно малой и ограниченной последовательности -- бесконечно малая последовательность.
        \end{enumerate}
    
    \begin{defi} Последовательность $ \{ x_ n \} $ называется бесконечно большой, 
        если $\displaystyle \lim_{n \rightarrow \infty} x_n = \infty$, т.е. 
        $\forall \eps > 0\ \exists N = N(\eps):|x_n| > \eps\ \forall n > N(\eps).$
    \end{defi}
    \textbf{Свойства б.б. последовательностей.}
        \begin{enumerate}
            \item Сумма бесконечно больших последовательностей одного знака есть бесконечно большая последовательность того же знака.
            \item Сумма бесконечно большой и ограниченной последовательностей есть бесконечно большая последовательность.
            \item Произведение бесконечно больших последовательностей есть бесконечно большая последовательность. Произведение бесконечно большой последовательности на константу есть бесконечно большая последовательность.
        \end{enumerate}

    \subsection*{Арифметические свойства пределов.}
    Если существует $\lim x_n = a$ и $\lim y_n = b$, то
        \begin{enumerate}
            \item $\lim (x_n \pm y_n) = \lim (x_n) \pm \lim(y_n) = a \pm b$.

                \begin{proof}[Доказательство]
                    $\exists\alpha_n = x_n - a$ -- б. м. и
                    $\exists\beta_n = y_n - b$ -- б. м.

                    Тогда внимательно посмотрим на $z_n = x_n + y_n$. Осознаем, что \linebreak $\{z_n - a - b\}$ -- бесконечно малая последовательность ($z_n - a - b = \alpha_n + \beta_n$ -- конечная сумма б.м. $|z_n - a - b| \le |x_n - a| + |y_n - b| = |\alpha_n| + |\beta_n|$).

                    По свойству (1) б.м.: $\{ z_n - a - b \}$ - беск. малая $\Leftrightarrow$ $\lim z_n = a + b$.
                \end{proof}

            \item $\lim (x_n y_n) = \lim (x_n) \lim (y_n) = a \cdot b$.

                    \begin{proof}[Доказательство 2]
                        $x_n = \alpha_n + a$  и
                        $y_n = \beta_n + b$, где $\alpha,\ \beta$ - беск. малые.

                        Тогда $\displaystyle \lim x_n y_n = \lim (\alpha_n + a)(\beta_n + b) = \lim (\alpha_n\beta_n + a\beta + b\alpha + ab) = 0 + 0 + 0 + ab = ab$.
                    \end{proof}

            \item $\exists \lim (\frac{x_n}{y_n}) = \frac{\lim (x_n)}{\lim (y_n)} = \frac{a}{b} $, если $\forall n:\ y_n \ne 0,\ b \ne 0$.

            \begin{proof}[Доказательство 3]
                $x_n = \alpha_n + a$  и
                $y_n = \beta_n + b$, где $\alpha,\ \beta$ - беск. малые.

                Тогда $\displaystyle \lim (\frac{x_n}{y_n}) = \
                lim (\frac{\alpha_n + a}{\beta_n + b}) = 
                \lim (\frac{a}{\beta_n + b} + \alpha \frac{1}{\beta_n + b}) = \lim (\frac{a}{\beta_n + b})$, но так как $\lim (\frac{a}{\beta_n + b} - \frac{a}{b}) = 0$, то $\lim \frac{a}{\beta_n + b} = \frac{a}{b}$.
            \end{proof}
        \end{enumerate}
\end{document}