\section*{Задача 29}
\begin{ther}
[Коши о промежуточном значении]
Пусть дана непрерывная функция на отрезке
$f\in C{\bigl (}[a,b]{\bigr )}$. Пусть также $f(a)\neq f(b)$,
и без ограничения общности предположим, что $f(a)=A<B=f(b)$.
Тогда для любого $C\in [A,B]$ существует $c\in [a,b]$ такое, что $f(c)=C$.
\end{ther}
\begin{proof}
	Рассмотрим функцию ${\displaystyle g(x)=f(x)-C.}$ Она непрерывна на отрезке
	${\displaystyle [a,b]}$  и ${\displaystyle g(a)<0}$  ${\displaystyle g(b)>0.}$
	Покажем, что существует такая точка $c\in [a,b]$, что ${\displaystyle g(c)=0.}$
	Разделим отрезок $[a,b]$ точкой $x_{0}$ на два равных по длине
	отрезка, тогда либо ${\displaystyle g(x_{0})=0}$ и нужная
	точка ${\displaystyle c=x_{0}}$ найдена, либо $g(x_{0})\neq 0$
	и тогда на концах одного из полученных промежутков функция $g(x)$
	принимает значения разных знаков (на левом конце меньше нуля, на правом больше).
	Обозначив полученный отрезок $[a_{1},b_{1}]$, разделим
	его снова на два равных по длине отрезка и т.д. Тогда, либо через конечное число
	шагов придем к искомой точке c, либо получим последовательность
	вложенных отрезков ${\displaystyle [a_{n},b_{n}]}$ по
	длине стремящихся к нулю и таких, что
	${\displaystyle g(a_{n})<0<g(b_{n}).}$
	Пусть c - общая точка всех отрезков (согласно принципу Кантора, она существует и единственна)
	 ${\displaystyle [a_{n},b_{n}]} $
	  ${\displaystyle n=1,2,...}$ Тогда
		$c=\lim a_{n}=\lim b_{n}$, и в силу непрерывности функции ${\displaystyle g(x):}$
 $g(c)=\lim g(a_{n})=\lim g(b_{n})$.
	Поскольку
$\lim g(a_{n})\leq 0\leq \lim g(b_{n})$,
	получим, что ${\displaystyle g(c)=0.}$
\end{proof}
\begin{ther}
[Коши о нуле непрерывной функции]
Если функция непрерывна на некотором отрезке и на концах этого отрезка принимает
значения противоположных знаков, то существует точка, в которой она равна нулю.
Данная теорема следует из предыдущей.
\end{ther}
\section*{Задача}
Докажите, что уравнение $x^{3} + 5x - 1 = 0$ имеет единственный корень $c$ и
найдите его с точностью 0.01 методом половинного деления.
\subsection*{Решение}
Найдем производную функции $f(x) = x^{3} + 5x - 1$. Она равна $3x^{2}+5$. Легко
убедиться, что не существует решения уравнения $3x^{2}+5=0$, а это означает, что
функция монотонна. Тогда получается, что у исходного уравнения может быть только
одно решение. Заметим, что $f(a) = f(0) = -1$, $f(b) = f(1) = 5$. Тогда по теореме Коши,
корень находится между числами 0 и 1. Будем искать его методом дихотомии.
$c_{1} = \frac{0+1}{2} = 0.5 $. $f(c_{1}) = 1.875$. $f(a)\cdot f(c1) < 0 $ , значит
корень лежит на отрезке $(0,0.5)$. $c_{2} = \frac{0.5}{2} = 0.25$. $f(c_{2}) = 0.29$.
$c_{3} = \frac{0.25}{2} = 0.125$. $f(c3)\cdot f(0.25) < 0 \implies $ корень лежит
между 0.125 и 0.25. Повторяем аналогичные действия, пока $f(c_{n})$ не станет меньше
0.01.
