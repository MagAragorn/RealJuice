Дайте определение производных и дифференциалов высших порядков. Докажите
формулу Лейбница. Найдите $y^{(10)}(0)$, если $y=\frac{1}{4y^2-8y+3}$

\subsection*{Решение}
\begin{definition}
    Производной $n$-го порядка называется рекуррентное соотношение:
        $$
        f^{(n)}(x) = (f^{(n-1)}(x))\drv; \,\,\,
        f^{(0)}(x) = f(x)
        $$
\end{definition}

\begin{definition}
    Дифференциалом $n$-го порядка называется рекуррентное соотношение:
    $$
        d^{(n)} = f^{(n)}dx^{n}
    $$
\end{definition}
\begin{proof}
    Воспользуемся индукцией.

    \par\underline{База:} $n=2$, тогда $d^2f = d(df) =
    d(f\drv x dx) = d(f\drv)\cdot dx = (f\drv(x))\drv\cdot dx\cdot dx =
    f^{(2)}(x)\cdot dx^2$.

    \par\underline{Переход:}
    $d^{n}f = d(d^{n-1}f) = (\text{по предположению индукции}) =
    f^{(n)}(x)dx^n$
\end{proof}

\begin{definition}
Формула Лейбница для производных произведений функций $n$-го порядка:
    Пусть $f(x)$ и $g(x)$ \textemdash $n$ раз дифференцируемые функции,
    тогда
    $$
        (f\cdot g)^{(n)} = \sum\limits_{k=0}^{n} {n \choose k}
        f^{(n-k)}\cdot g^{(k)}
    $$
\end{definition}
\begin{proof}
    \underline{База:} $n=1$, $(f\cdot g)\drv = f\drv g + fg\drv = {1
    \choose 0}\cdot f^{(1)}g^{(0)} + {1 \choose 1}\cdot f^{(0)}g^{(1)} =
    \sum\limits_{k=0}^{1} {n \choose k} f^{(n-k)}\cdot g^{(k)}$

    \underline{Переход:} рассмотрим $n+1$ производную:
    $(fg)^{(n+1)} =
    ( \sum\limits_{k=0}^{n} {n \choose k} f^{(n-k)}\cdot g^{(k)})\drv =
    \sum\limits_{k=0}^{n} {n \choose
    k}(f^{(n-k+1)}g^{(k)}+f^{(n-k)}g^{(k+1)}) =
    \sum\limits_{k=0}^{n} {n \choose k} f^{(n-k+1)}\cdot g^{(k)}
    +
    \sum\limits_{k=0}^{n} {n \choose k} f^{(n-k)}\cdot g^{(k+1)}
    $
    Докажем, что эти две суммы можно преобразовать в одну сумму
    произведений. Сменим индекс суммирования во второй сумме: $k=m-1$, в
    первой $k=m$. Тогда элементы первой суммы будут иметь вид
    ${n \choose m} f^{(n-m+1)}\cdot g^{(m)}$, а элементы второй \textemdash
    ${n \choose m-1} f^{(n-(m-1))}\cdot g^{((m-1)+1)} =
    {n \choose m-1} f^{(n-m+1))}\cdot g^{(m)}.
    $
    Из комбинаторики известно, что ${n \choose m} + {n \choose m-1} = {n+1
    \choose m}$, тогда сумма таких элементов имеет вид:

    ${n \choose m} f^{(n-m+1)}\cdot g^{(m)} + {n \choose m-1}
    f^{(n-(m-1))}\cdot g^{((m-1)+1)} =
    {n+1 \choose m} f^{(n-m+1))}\cdot g^{(m)}.$

    При изменении $m$ от 1 до $n$ такое объединение даёт все слагаемые
    обеих сумм, кроме:
    \begin{itemize}
        \item члена $i = 0$ в первой сумме, равного ${n \choose
            0}f^{(n-0+1)}g^{(0)} = f^{(n+1)}g^{(0)}$
        \item члена $i = n$ вло второй сумме, равного ${n \choose
            n}f^{(n-n)}g^{(n+1)} = f^{(0)}g^{(n+1)}$
    \end{itemize}
    В итоге получаем, что $(fg)^{(n+1)} =
    f^{(n+1)}g^{(0)}
    +
    \sum\limits_{m=1}^{n}{n+1 \choose m}
    f^{(n+1-m))}\cdot g^{(m)}.
    +
    f^{(0)}g^{(n+1)}
    $
    Обратная замена $m=k$, тогда
    $
    \sum\limits_{k=0}^{n+1}{n+1 \choose k} f^{(n+1-k))}\cdot g^{(k)}
    $
\end{proof}
