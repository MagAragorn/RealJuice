Дайте определение равномозных множеств, счетного множества. \textit{Докажите}, что счетное объединение счетных множеств счетно. Покажите, что $(0,1] $ равномощно $ [0, 1]$.
\subsection{}
Множества называются равномощными, если существует биекция из одного множества в другое.
\subsection{}
Множество называется счетным, если оно равномощно $\mathbb{N}$
\subsection{}
Счетное объединение -- объединение счетного множества множеств.\\
Пусть есть счетное множество множеств $A_0, A_1, A_2 ... $. Тогда составим такую таблицу, что в i-й строке на j-м месте будет стоять элемент i-го множества(множеств счетное количество -- можем занумеровать их) с номером j(все множества счетны -- можем занумеровать элементы внутри них). Этот элемент обозначим за $a_{ij}$. Далее выделим в этой таблице <<диагонали>>, т.е. последовательности элементов, сумма индексов которых равна. Нулевая диагональ -- левый верхний элемент $a_{00}$; первая диагональ, элементы $a_{01}$ и $a_{10}$; вторая -- $a_{02}$, $a_{11}$, $a_{20}$... Далее будем строить последовательность всех элементов всех множеств. Будем брать диагонали по возрастанию суммы индексов в них и дописывать в конец последовательности(в самом начале пустой) числа этой диагонали в порядке возрастания номера строки(= убывания номера столбца). При этом в случае, если очередной выписываемый элемент уже встретился и был выписан ранее, то не будем его выписывать.\\
Ясно, что в итоге получится счетное множество, т.к. счетно как каждое из множеств, так и множество всех множеств. Ч.т.д.
\subsection{}
Из $(0;1]$ выделим в множество $A$ бесконечную убывающую последовательность: $\{\frac{1}{1}, \frac{1}{2},\frac{1}{3}, ...\}$. Останется $(0;1]\setminus A$. Из $[0;1]$ выделим в множество $A`$ сперва 0, а затем точно такую же последовательность, как выделили из $(0;1]$. Останется $[0;1]\setminus A`$. Заметим, что после выделения $A$ и $A`$ и там и там остались одинаковые множества. Биекция между ними очевидна. Осталось провести биекцию между выделенными последовательностями:
$$A = \{\frac{1}{1}, \frac{1}{2}, \frac{1}{3}, \frac{1}{4}, ...\},$$
$$A` = \{0, \frac{1}{1}, \frac{1}{2}, \frac{1}{3}, ...\}.$$
Просто сопоставим элементы этих множеств таким образом: $\frac{1}{1}\leftrightarrow 0, \frac{1}{2}\leftrightarrow \frac{1}{1}, \frac{1}{3}\leftrightarrow \frac{1}{2}, \frac{1}{4}\leftrightarrow \frac{1}{3} ...$
