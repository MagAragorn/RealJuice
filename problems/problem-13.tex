\begin{definition}[Подпоследовательность]
	Пусть есть любая возрастающая последовательность $f$, и последовательность $a$. Тогда последовательность $a_{f_0}, a_{f_1}, a_{f_2}, \cdots$ называется подпоследовательностью.
\end{definition}

\begin{definition}[Частичный предел]
	Это предел какой-то подпоследовательности.
\end{definition}


\begin{definition}[Верхний предел]
	\[\overline{ \lim_{x \rightarrow \infty}} = \sup \{ \forall f: \exists \lim_{i \rightarrow \infty} x_{f_i}, \lim_{i \rightarrow \infty} x_{f_i}\}\]
	Говоря словами это точная верхняя грань всех частичных пределов.
\end{definition}


\begin{definition}[Нижний предел]
	\[\overline{ \lim_{x \rightarrow \infty}} = \inf \{ \forall f: \exists \lim_{i \rightarrow \infty} x_{f_i}, \lim_{i \rightarrow \infty} x_{f_i}\}\]
	Говоря словами это точная нижняя грань всех частичных пределов.
\end{definition}


\begin{examples}
	Последовательность $a_i = \frac1i$. Её верхний предел равен нижнему и равен 0. (В самом деле не трудно доказать, что если у последовательности сущетвует предел, то любой её частичный предел ему равен.)
\end{examples}


\begin{examples}
	\[x_n = \frac{(-1)^n n + 1}{n + 1}\]
	Если в подпоследовательности конечное количество элементов с четными индексами, то с некоторого момента в ней будут только элементы с нечетными индексами, поэтому при подсчете предела мы можем считать что все элементы с нечетными индексами, аналогично с четными, тогда необходимо рассмтреть всего 3 случая.

	\begin{itemize}
		\item \textbf{Есть конечное число элементов с четными индексами.} Тогда предел равен \[\lim_{n \rightarrow \infty} \frac{-n + 1}{n + 1} = -1\]
		\item \textbf{Есть конечное число элементов с нечетными индексами.} Тогда предел равен \[\lim_{n \rightarrow \infty} \frac{n + 1}{n + 1} = 1\]
		\item \textbf{Есть бесконечное число элементов с четными индексами и бесконечное число с нечетными.} Предела не существует, иначе возьмем окрестность размера меньше одной десятой, но при четных индексах значение ровно 1, а при нечетных $\frac{1 - n}{n + 1} = -1 + \frac{2}{n + 1}$ разность которых возрастает (и когда-то станет больше $0.1$), противоречие.
	\end{itemize}
\end{examples}
