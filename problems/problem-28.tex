
\begin{ther}
[Теорема Вейерштрасса о достижимости точных граней	функцией, непрерывной на отрезке]
Если функция f непрерывна на отрезке [a,b],
то она ограничена на нем и притом достигает своих минимального и максимального значений,
т.е. существуют $ x_m,\;x_M\in[a,\;b]$ такие,
что $f(x_{m})\leq f(x)\leq f(x_{M})$
для всех $x\in [a,b]$.
\begin{proof}
В силу полноты действительных чисел существует (конечная или бесконечная) точная
верхняя грань $M=\sup _{{x\in [a,b]}}f(x)$.
Поскольку M — точная верхняя грань, существует последовательность
$x_{n}$ такая, что $\lim f(x_{n})=M$.
По теореме Больцано — Вейерштрасса из ограниченной последовательности
$x_{n}$ можно выделить сходящуюся подпоследовательность $x_{{n_{k}}}$,
предел которой (назовем его $x_M$) также принадлежит отрезку
[a,b]. В силу непрерывности функции f имеем
$f(x_{M})=\lim f(x_{{n_{k}}})$, но с другой стороны $\lim f(x_{{n_{k}}})=\lim f(x_{n})=M$.
Таким образом, точная верхняя грань  M конечна и достигается в точке $x_M$.
Для нижней грани доказательство аналогично.
\begin{ther}
Теорема Вейерштрасса об ограниченности функции, непрерывной на отрезке.
\end{ther}
\begin{proof}
	Пусть f неограниченна на отрезке [a,b], тогда :
	$\forall c>0 \exists x_{c} \in [a,b]:|f(x_{c})|>c$\\
	$c=1 \exists xc \in [a,b]:|f(x_{1})|>1$\\
	...\\
	$c=n \exists xc \in [a,b]:|f(x_{n})|>n$\\
	Получим последовательность $x_{n} \subset [a,b]$ ,
	то есть последовательность $x_{n}$ ограниченная
Отсюда по теореме Больцано-Вейерштрасса из нее можно выделить подпоследовательность,
 которая сходится к точке $\xi$ , то есть $\lim_{k\to \infty}{x_{n_{k}}} = f(\xi)$.
 $\xi \in [a,b]$ по свойству пределов в форме неравенств. Но по условию функция f
 непрерывна в точке $\xi$ и тогда по определению непрерывности точки по Гейне:
 $\lim_{k\to \infty}{f(x_{n_{k}})} = f(\xi)$. С другой стороны $|f(x_{n_{k}})|
 > n_{k}, n_{k} \geq k \implies$ $\lim_{k\to \infty}{f(x_{n_{k}})} = \infty$
 А это противоречит единственности предела.
\end{proof}
\end{proof}
\end{ther}
