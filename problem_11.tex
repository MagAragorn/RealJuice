\documentclass[a4paper,12pt]{article}

\usepackage[T1,T2A]{fontenc}        % Кодировки шрифтов
\usepackage[utf8]{inputenc}         % Кодировка текста
\usepackage[english,russian]{babel} % Подключение поддержки языков

\usepackage{amsthm}                 % Оформление теорем
\usepackage{amstext}                % Текстовые вставки в формулы
\usepackage{amsfonts}               % Математические шрифты

\newtheorem*{ther}{Теорема}
\newtheorem*{defi}{Определение}

\newcommand{\eps}{\varepsilon}

\begin{document}

    \section*{Билет 11}

    \begin{ther}
        Если последователность монотонна и ограничена, то она имеет конечный предел.
    \end{ther}

    \begin{ther}[Теорема Вейерштрасса (о пределе монотонной ограниченной последовательности)]

        Если последовательность является возрастающей и ограниченной сверху, то: $\lim\limits_{x \to \infty} x_n = \sup {x_n}$.

        Аналогично для убывающей и ограниченной снизу последовательности: $\lim\limits_{x \to \infty} x_n = \inf {x_n}$.
    \end{ther}

    \begin{proof}
        Докажем теорему для монотонной возрастающей последовательности $\left\{x_n\right\}$. Докажем, что точная верхняя граница $a = \sup{x_n}$ для последовательности и будет ее пределом.

        Действительно, по определению точной верхней границы: 
        $\forall n\ x_n \leq a$.
        Кроме того, какое бы ни взять число $\eps > 0$, найдется такой номер $N$, что $x_N > a - \eps$.
        Так как последовательность монотонна, то при $n > N: x_n \geq x_N$, а значит, и $x_n > a - \eps$ и выполняются неравенства: $0 \leq a - x_n < \eps \vee \left | x_n - a \right | <\eps$
         откуда и следует, что $\lim\limits_{n \to \infty} x_n = a$.
    \end{proof}
\end{document}