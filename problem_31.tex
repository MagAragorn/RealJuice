\documentclass[a4paper,12pt]{article}

\usepackage[T1,T2A]{fontenc}        % Кодировки шрифтов
\usepackage[utf8]{inputenc}         % Кодировка текста
\usepackage[english,russian]{babel} % Подключение поддержки языков

\usepackage{amsthm}                 % Оформление теорем
\usepackage{amsmath}                 % Оформление теорем
\usepackage{amstext}                % Текстовые вставки в формулы
\usepackage{amsfonts}               % Математические шрифты

\newtheorem*{ther}{Теорема}
\newtheorem*{defi}{Определение}

\newcommand{\eps}{\varepsilon}
\newcommand{\drv}{^{\prime}}

\begin{document}

    \section*{Билет 31}
    Дайте определение дифференцируемой функции и дифференциала. Укажите
    геометрический смысл дифференциала.

    \subsection*{Решение}
    \begin{defi}
        Функция $f:E\rightarrow \mathbb{R}$ называется
        \textbf{дифференцируемой} в точке $x_0\in E$, если существует такая
        линейная (относительно приращения $x-x_0$ аргумента) функция $A\cdot
        (x-x_0)$, что приращение $f(x) - f(x_0)$ функции $f$ представляется в
        виде $f(x)-f(x_0) = A \cdot (x-x_0) + o(x-x_0)$ при $E \ni x \to x_0$.
    \end{defi}

    \begin{defi}
        Линейная функция $A\cdot(x-x_0)$ называется дифференциалом функции $f$
        в точке $x_0$. Из предыдущего определения
        $\lim\limits_{E\ni x \to x_0}\frac{f(x)-f(x_0)}{x-x_0}
        =
        \lim\limits_{E\ni x \to x_0}\big(A + \frac{o(x-x_0)}{x-x_0} \big) = A$;
        в силу единственности предела (билет 8) число $A$ определено
        однозначно.
    \end{defi}

    \begin{defi}
        Геометрический смысл: дифференциал функции $f$ есть приращение ординаты
        касательной в графику $f$ в данной точке.
    \end{defi}
\end{document}
